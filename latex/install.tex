
\section{Installing}
\label{install}

The installation for TauP under \textsc{Unix} is quite simple. And with Java's 
platform independence, the package should be usable on a Mac or Windows
machine.

\subsection{Unix}

\begin{enumerate}
\item Install a Java 1.1 or better virtual machine. If your system already has Java 1.1
or better installed then you can skip to the next step. You can test this
with ``java -version''. If it isn't there or the version is less than 1.1
you need to get and install Java.

If you have a Sun Solaris
workstation, you need only down load the version from JavaSoft. Point
your browser to http://www.javasoft.com/products/ and download either
the Java Development Kit or the Java Runtime Environment.

The Java Runtime Environment (jre)
is the smaller of the two, only allowing you to run java applications and applets.
The Java Development Kit (jdk) is larger but allows you
to compile and run java programs. 
Unless space is at a
premium, I suggest getting the jdk. A quick test to see if you have
the jdk is to see if javac, the java compiler is installed. The command
``which javac'' or just ``javac'' should tell you. 

There are ports of Java for many other operating systems, and as long as
they are a Java 1.1 or better compatible implementation, the tools 
should execute correctly.
JavaSoft maintains a list of these ports at\newline
http://www.javasoft.com/cgi-bin/java-ports.cgi.

Just follow the instructions that come with your Java distribution.

\item Download TauP.X.X.X.tar.gz or TauP.X.X.X.jar. Make sure to get the most recent version, replacing the X's  in the file name. They can be found at

\texttt{http://www.seis.sc.edu}

\item Unpack the distribution.
\texttt{\newline gunzip TauP.X.X.X.tar.gz\newline tar -xvf TauP.X.X.X.tar\newline}
or
\texttt{\newline jar -xvf TauP.X.X.X.jar\newline}
This will create a directory called TauP. Inside
will be the following:
\begin{center}
\begin{tabular}{lp{4in}}
README & getting started information \\
License & license information, free for non-commercial \\
taup.jar & the jar file with all the classes \\
taup.html & a simple web page that loads a rudimentary applet to use the TauP
package. Be warned that this is not really meant to be used over the Internet
as the download time for 1.5Mb may be too great and most browsers do not yet
support the 1.1 version of Java. \\
exampleProperties & example properties file \\
HISTORY & change log \\
bin & a directory with wrapper scripts appropriate for 
\textsc{Unix} installations\\
doc & a directory with Postscript and pdf versions of this manual. \\
html & a directory with the javadoc output from
the source code, mainly useful
for writing new java programs that use the TauP package. \\
jacl & a directory with example Jacl scripts for accessing the TauP package. \\
native & a directory with a C library and example program that use the
Java Native Interface, providing a basic interface between C programs
and the TauP package. \\
Maple & a directory with Maple scripts showing the time and distance equations
used.
\end{tabular}
\end{center}

\item Put the taup.jar file someplace. It really doesn't matter where, although
a central place might make administration easier, /usr/local/classes or 
/usr/local/lib are good choices. If you don't have superuser privileges
then your home directory is fine. 

For java 2 (a.k.a java 1.2) you can install taup.jar as a \textit{standard extension} by placing the jar file into the jre/lib/ext subdirectory of your java installation. If you do this then you do not need to add it to your CLASSPATH and can skip the next step.

\item Add the location of taup.jar to your CLASSPATH environment variable.
This should be done in your .cshrc or .login. For instance, if you put taup.jar in
/usr/local/classes, then you could set the CLASSPATH to be:
\begin{verbatim}
setenv CLASSPATH /usr/local/classes/taup.jar:/usr/local/jdk1.1.6/lib/classes.zip
\end{verbatim} 

This should be all one line, of course.
The java virtual machine also uses the CLASSPATH environment variable to find
its class files, so make sure this is set up correctly.

\item Put the wrapper scripts in a directory referenced by your PATH environment
variable, /usr/local/bin for instance. These wrapper 
scripts are not essential, but they cut down on 
typing. They are in the bin directory of the distribution 
and are simple \textsc{Unix} sh scripts.
Of course, they will only work on \textsc{Unix}.

\item Lastly, you may need to either source your .login and .cshrc files or
execute the \texttt{rehash} command to make the shell reevaluate the 
contents of your PATH.

\end{enumerate}

That's it. If you have problems or encounter bugs, please mail them to me.
Please try to be as specific as possible. I am also interested in ideas for
additional features that might make this a more useful program. 
Of course, I can make no promises,
but I would be glad to hear about them.

I can be reached via email at \textit{crotwell@seis.sc.edu}.

\subsection{MacOS}

Please read the Unix instructions, since much of it is applicable to all platforms.

For MacOS users, you can add taup.jar to your classpath by putting it in the
MRJclasses folder which is located in the MRJ libraries folder within the Extensions folder within the System Folder. You may also wish to change the file type and creator to ``ZIP '' and
``java'' in order to fix the icon. Note that ZIP has a space at the end. This can be done with the DataViz FileViewer application that is, or at least was, bundled with the system. Also note that you must haveMacOS~8 or better in order to run Apple's java 1.1 compatible java, MRJ 2.1. Lastly, JBindary is the MacOS version of the java command. With it you can create double-clickable applications that accomplish the same thing as the sh scripts under unix. If you do not have MRJ installed, or would like to get a more recent version, it can be downloaded from Apple's web site, \verb+http://www.apple.com/java+.

\subsection{Windows}

Please read the Unix instructions, since much of it is applicable to all platforms.

For Java 1.1, you need to add taup.jar to your CLASSPATH. This should likely 
be done in you autoexec.bat file for window 95 or 98 and in the Control Panel 
under windows NT, start the Control Panel, select System, then click the 
Environment tab. The other difference with \textsc{Unix} is that the separator is a semicolon and the forward slashes should be replaced with backslashes.

For Java 1.2, aka Java 2, you can skip the CLASSPATH step by putting the taup.jar file into the \verb+jre\lib\ext+ directory of your java installation.
 
In either case you will likely want to add the bat files that start up the 
tools to your PATH. You can either put them in and existing directory 
referenced in you PATH, or add the distribution bat directory to the 
PATH environment variable. I have put the bat files in the bat directory to 
 keep them separate from the \textsc{Unix} sh scripts, and so you may wish 
to delete bin and rename bat to bin.