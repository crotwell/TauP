
\section{Examples} \label{exampleCreate}

Here is a walk through of a use of the tools on a \textsc{Unix} system. 

\subsection{Velocity Model Files} \label{exampleND} 
First, we want to create a model. There are several models contained within
the TauP distribution, but for completeness we will create a new one 
from scratch.

A very simple model file might look like this:

\begin{verbatim}
0.0     5.0  3.0   2.7
20      5.0  3.0   2.7
20      6.5  3.7   2.9
33      6.5  3.7   2.9
mantle
33      7.8  4.4   3.3
410     8.9  4.7   3.5
410     9.1  4.9   3.7
670    10.2  5.5   4.0
670    10.7  5.9   4.4
1000   11.5  6.4   4.6
2000   12.9  6.9   5.1
2891   13.7  7.2   5.6
outer-core
2891    8.0  0.0   9.9
3500    9.0  0.0  10.8
5149.5 10.3  0.0  12.2
inner-core
5149.5 11    3.5  12.7
5500   11.1  3.6  12.9
6371   11.3  3.7  13
\end{verbatim}

Note that we have chosen the ``named discontinuities'' format so
that we could specify the location of the major boundaries.
The file consists of two types of lines, those that specify velocity at a depth,
and those that specify the name of a discontinuity. See section \ref{velfile}
for more details.

\subsection{Creating the Model}

If we put this into a file called ``simpleMod.nd'' then we can run 
\texttt{taup\_create} to create a model sampling. We use the 
\texttt{-verbose} option to get some additional output. In particular,
it outputs the radius to the surface for this model. Having an incorrect
radius, which could happen for instance 
if the last line of the model file was lost,
will generate incorrect times for phases that otherwise appear fine.
This can be a difficult error to track down after the fact because
there is nothing wrong with the model, it is just not what was intended.
As they say, ``garbage in, garbage out.''

\begin{verbatim}
piglet 10>taup_create -nd simpleMod.nd -verbose
TauP_Create starting...
filename =./simpleMod.nd
Done reading velocity model.
Radius of model simpleMod is 6371.0
Parameters are:
taup.create.minDeltaP = 0.1 sec / radian
taup.create.maxDeltaP = 8.0 sec / radian
taup.create.maxDepthInterval = 115.0 kilometers
taup.create.maxRangeInterval = 1.75 degrees
taup.create.maxInterpError = 0.03 seconds
taup.create.allowInnerCoreS = true
Slow model time=39714 801 P layers,907 S layers
T model time=7480
Done Saving ./simpleMod.taup
Done!
Done!
piglet 11>ls
simpleMod.nd     simpleMod.taup
\end{verbatim}

The file \texttt{simpleMod.taup} contains all of the information about the
model. This process needs to be done only once for each velocity model.
The times appearing in the output are in milliseconds, and do not reflect the
startup time for Java.

\subsection{Travel Times}

Now that we have the model sampled, computing travel times is easy.
We will use taup\_time to get the travel times for some familiar phases, 
P, S, PcP, ScS, SKS, sS, and SS 
in our simple model for a 143.2 kilometer deep 
source and at a distance of 75 degrees. We use the ``-mod'' command line
flag to specify the model, and then do the rest after it starts.

First taup\_time reads a standard Java Properties file, ``.taup'', 
that it finds in my home directory. See section \ref{properties} for 
more details.
If there are phases you are interested in frequently, or model you use often,
or source depth, then you can put them in this file as your defaults.
Then we enter a depth for the source, 143.2 kilometers, using the \texttt{h}
option. By default, the model is for a surface source.

Some phase names have been read in from the file, but
we want to specify our own phase list, so we use the `c' option
to clear the phases and are prompted to enter the new phases. Enter them 
separated by commas or spaces. After that we just need to enter the distance,
75 degrees. The arrivals are printed as distance, depth, phase name, time and then
ray parameter. 
The last two entries represent a ``purists'' view of the distance and phase name. For instance, PKKP travels the long way around the earth, and so the 
true distance traveled is not the event to station distance. The purist's 
view of the name is to show the difference between the true depths of discontinuities and the depth specified in the phase name. For instance, 
Pv400P in our simple model is really a reflection off of the discontinuity 
at 410 kilometers depth. The purist's name reflects this and is preceded 
by an asterisk to make the difference easier to notice. 
The distance is repeated to make it easier to parse the output
from within scripts.

\begin{verbatim}
piglet 4>taup_time -mod simpleMod
Enter:
h for new depth
r to recalculate
p to append phases, 
c to clear phases
l to list phases
s for new station lat lon
e for new event lat lon
a for new azimuth
b for new back azimuth
m for new model or 
q to quit.

Enter Distance or Option [hrpclseabmq]: h
Enter Depth: 143.2
Enter Distance or Option [hrpclseabmq]: c
Enter phases (ie P,p,PcP,S): P,S,PcP,ScS,SKS,sS,SS,PKKP
Enter Distance or Option [hrpclseabmq]: 75

Model: simpleMod
Distance   Depth   Phase   Travel    Ray Param   Purist    Purist
  (deg)     (km)   Name    Time (s)  p (s/deg)  Distance   Name
----------------------------------------------------------------
    75.0   143.2   P        686.33     5.721       75.0  = P    
    75.0   143.2   PcP      700.51     4.312       75.0  = PcP  
    75.0   143.2   S       1263.17    11.040       75.0  = S    
    75.0   143.2   SKS     1293.35     7.283       75.0  = SKS  
    75.0   143.2   ScS     1298.74     8.135       75.0  = ScS  
    75.0   143.2   sS      1326.73    11.152       75.0  = sS   
    75.0   143.2   SS      1571.74    14.640       75.0  = SS   

Enter Distance or Option [hrpclseabmq]: q
\end{verbatim}

We could also have done this same example by just using the command line
options.

\begin{verbatim}
piglet 5>taup_time -mod simpleMod -h 143.2 -deg 75 -ph P,S,PcP,ScS,SKS,sS,SS,PKKP

Model: simpleMod
Distance   Depth   Phase   Travel    Ray Param   Purist    Purist
  (deg)     (km)   Name    Time (s)  p (s/deg)  Distance   Name
----------------------------------------------------------------
    75.0   143.2   P        686.33     5.721       75.0  = P    
    75.0   143.2   PcP      700.51     4.312       75.0  = PcP  
    75.0   143.2   S       1263.17    11.040       75.0  = S    
    75.0   143.2   SKS     1293.35     7.283       75.0  = SKS  
    75.0   143.2   ScS     1298.74     8.135       75.0  = ScS  
    75.0   143.2   sS      1326.73    11.152       75.0  = sS   
    75.0   143.2   SS      1571.74    14.640       75.0  = SS
\end{verbatim}

\subsection{Pierce Points}

Now, where are the turning points for these rays? We can run taup\_pierce with
the ``-turn'' flag and
find out. Lets specify the parameters on the command line.
The output is distance in degrees followed by depth in kilometers. 
Note that SS has two turning points. 

\begin{verbatim}
piglet 7>taup_pierce -mod simpleMod -h 143.2 -deg 75 \
? -ph P,S,PcP,ScS,SKS,sS,SS,PKKP -turn
> P at 686.33 seconds at 75.0 degrees for a 143.2 km deep source in the simpleMod model.
   37.23 2110.32
> S at 1263.17 seconds at 75.0 degrees for a 143.2 km deep source in the simpleMod model.
   37.20 2005.24
> PcP at 700.51 seconds at 75.0 degrees for a 143.2 km deep source in the simpleMod model.
   37.30 2891.00
> ScS at 1298.74 seconds at 75.0 degrees for a 143.2 km deep source in the simpleMod model.
   37.29 2891.00
> SKS at 1293.35 seconds at 75.0 degrees for a 143.2 km deep source in the simpleMod model.
   37.31 2975.11
> sS at 1326.73 seconds at 75.0 degrees for a 143.2 km deep source in the simpleMod model.
   37.81 1971.17
> SS at 1571.74 seconds at 75.0 degrees for a 143.2 km deep source in the simpleMod model.
   18.09 1001.76
   56.03 1001.76
\end{verbatim}

\subsection{Path}

Perhaps now we should make a plot of the
paths. Lets use only command line options and send the output to the file 
``simpleModPaths.gmt'' instead of the default ``taup\_path.gmt''.

\begin{verbatim}
piglet 8>taup_path -mod simpleMod -h 143.2 -deg 75 \
? -ph P,S,PcP,ScS,SKS,sS,SS,PKKP \
? -o simpleModPaths.gmt -gmt
piglet 9>ls
simpleMod.taup       simpleMod.nd         simpleModPaths.gmt
piglet 10>sh simpleModPaths.gmt
piglet 11>ls
simpleMod.taup       simpleModPaths.ps
simpleMod.nd         simpleModPaths.gmt
\end{verbatim}

Now we have a Postscript file, simpleModPaths.ps, that we can look at or print.
Notice that we used the \texttt{-gmt} flag so that the output is a complete 
GMT script. If you don't use \texttt{-gmt}, then the output is just the XY 
points, which might be later used by another script.
Of course, this only works if you have GMT installed.

\subsection{Travel Time Curves}

If we want to see the travel time curves for these phases, we can do that using
taup\_curve. It works very similarly to taup\_path except that we don't need to
specify a distance.

\begin{verbatim}
piglet 12>taup_curve -mod simpleMod -h 143.2 -ph P,S,PcP,ScS,SKS,sS,SS,PKKP \
? -o simpleModCurves.gmt -gmt
piglet 13>ls
simpleModCurves.gmt   simpleMod.nd          simpleModPaths.gmt
simpleMod.taup        simpleModPaths.ps
piglet 14>sh simpleModCurves.gmt
piglet 15>ls
simpleMod.nd          simpleModCurves.gmt   simpleModPaths.gmt
simpleMod.taup        simpleModCurves.ps    simpleModPaths.ps
\end{verbatim}

Again we have a Postscript file to view. Both of these commands generate scripts
that are ok for a quick look, but you will almost certainly want to modify
them for any important use.
